%%%%%%%%%%%%%%%%%%%%%%%%%%%%%%%%%%%%%%%%%
% Arsclassica Article
% LaTeX Template
% Version 1.1 (1/8/17)
%
% This template has been downloaded from:
% http://www.LaTeXTemplates.com
%
% Original author:
% Lorenzo Pantieri (http://www.lorenzopantieri.net) with extensive modifications by:
% Vel (vel@latextemplates.com)
%
% License:
% CC BY-NC-SA 3.0 (http://creativecommons.org/licenses/by-nc-sa/3.0/)
%
%%%%%%%%%%%%%%%%%%%%%%%%%%%%%%%%%%%%%%%%%

%----------------------------------------------------------------------------------------
%	PACKAGES AND OTHER DOCUMENT CONFIGURATIONS
%----------------------------------------------------------------------------------------


\documentclass[
11pt, % Main document font size
a4paper, % Paper type, use 'letterpaper' for US Letter paper
oneside, % One page layout (no page indentation)
%twoside, % Two page layout (page indentation for binding and different headers)
headinclude,footinclude, % Extra spacing for the header and footer
BCOR5mm, % Binding correction
]{scrartcl}

\input{structure.tex} % Include the structure.tex file which specified the document structure and layout

\hyphenation{Fortran hy-phen-ation} % Specify custom hyphenation points in words with dashes where you would like hyphenation to occur, or alternatively, don't put any dashes in a word to stop hyphenation altogether

%----------------------------------------------------------------------------------------
%	TITLE AND AUTHOR(S)
%----------------------------------------------------------------------------------------

\title{\normalfont\spacedallcaps{Training Project Laboratory Report}} % The article title

%\subtitle{Subtitle} % Uncomment to display a subtitle

\author{\spacedlowsmallcaps{Qianhao Zhang}} % The article author(s) - author affiliations need to be specified in the AUTHOR AFFILIATIONS block

\date{} % An optional date to appear under the author(s)

%----------------------------------------------------------------------------------------

\begin{document}

%----------------------------------------------------------------------------------------
%	HEADERS
%----------------------------------------------------------------------------------------

\renewcommand{\sectionmark}[1]{\markright{\spacedlowsmallcaps{#1}}} % The header for all pages (oneside) or for even pages (twoside)
%\renewcommand{\subsectionmark}[1]{\markright{\thesubsection~#1}} % Uncomment when using the twoside option - this modifies the header on odd pages
\lehead{\mbox{\llap{\small\thepage\kern1em\color{halfgray} \vline}\color{halfgray}\hspace{0.5em}\rightmark\hfil}} % The header style

\pagestyle{scrheadings} % Enable the headers specified in this block

%----------------------------------------------------------------------------------------
%	TABLE OF CONTENTS & LISTS OF FIGURES AND TABLES
%----------------------------------------------------------------------------------------

\maketitle % Print the title/author/date block

\setcounter{tocdepth}{2} % Set the depth of the table of contents to show sections and subsections only

\tableofcontents % Print the table of contents

\listoffigures % Print the list of figures

\listoftables % Print the list of tables

%----------------------------------------------------------------------------------------
%	ABSTRACT
%----------------------------------------------------------------------------------------

\section*{Abstract} % This section will not appear in the table of contents due to the star (\section*)

This is a report for training project laboratory in electrical engineering infocommunication BSc course.
This laboratory consist of two parts, both will be covered by this report.
The first part is circuit realization and measurements around a radio direction 
finder receiver.
The second part is measurements, calculation and software tools realization in Space technology laboratory.

%----------------------------------------------------------------------------------------

\newpage % Start the article content on the second page, remove this if you have a longer abstract that goes onto the second page

 
%----------------------------------------------------------------------------------------
%	METHODS
%----------------------------------------------------------------------------------------

\section{Part1 - radio direction finder receiver}

From week 1 to 7 I was working on the radio direction finder receiver. Our task is to realize
the circuit following the schematic. The printed circuit board(PCB) is prepared in advance.

\subsection{Components}
\paragraph{surface mounted components}
All resistors, capacitors, inductors and transistor are packaged using surface mount technology(SMT).
Which makes them tricky to hand solder on the board. When to much soldering is applied
a short circuit could happen between pads underness components. A multimeter is very useful
for troubleshooting.

\paragraph{through hold components}
Audio jack, potentiometer and crystal oscillator are through hold components which provided
a firm connection to the PCB.



%------------------------------------------------

\subsection{Measurements}
The task is to measure the following parameters of the realized circuit and make test and measurement report based on the measurement results.

\begin{enumerate}
\item Bias DC voltages to the reference GND point on all pins of all semiconductors.
\item Voltage curve in time of the local oscillator output (emitter): peak-to-peak voltage and frequency.
\item Receiver audio (time domain) output signal on the AF output connector: 
variable resistor low, middle, high position: 
peak-to-peak voltage, frequency, curve. During this measurement, 
a single test transmitter will be run near to the receiver.
\end{enumerate}

\subsubsection{Bias DC voltage of transistors}
DC voltage is measured reference to ground. And potentiometer is set to
a low position during measurements.


\begin{table}[h]\centering
    \begin{tabular}{|l|l|}
        \cline{1-2}
        Transistor & Voltage \\
        \cline{1-2}
        Q1 & 807mV \\
        Q2 & 72.4mV \\
        Q3 & 2.79V \\
        Q4 & 1.94V \\
        Q5 & 96.5mV \\
        Q6 & 97.2mV \\
        Q7 & 96.9mV \\
        \cline{1-2}
    \end{tabular}
    \caption{Transistor voltage measurement}
\end{table}

\newpage

\subsubsection{Voltage curve in time domain of the local oscillator output}
In this measurement 10x probe gain is used. I used the quick measure function
on the oscilloscope.

\begin{figure}[h!]
    \includegraphics[width=\linewidth]{Figures/BUCK06.PNG}
    \caption{local oscillator output}
    \label{fig:local oscillator output}
\end{figure}


\newpage
\subsubsection{Receiver audio (time domain) output signal}

\begin{figure}[h!]
    \centering
    \subfloat[Low]{\includegraphics[width=0.4\linewidth]{Figures/BUCK08.PNG}} \quad
    \subfloat[Mid]{\includegraphics[width=0.4\linewidth]{Figures/BUCK09.PNG}} \\
    \subfloat[High]{\includegraphics[width=0.4\linewidth]{Figures/BUCK10.PNG}}
 
    \caption{Receiver audio output signal in time domain}
    \label{fig:Receiver audio output}
  \end{figure}

%----------------------------------------------------------------------------------------
%	Part2 - 58Ghz attenuation
%----------------------------------------------------------------------------------------
\newpage

\section{Part2 - 58Ghz attenuation}
From 8 to 14 week I worked for Space technology laboratory supervised by Dr. László Csurgai-Horváth.
Our topic is to research about oxygen and rain attenuation on 58GHz radio signal.

\subsection{Introduction on 58GHz band}
Signal around 60Ghz band has a unique advantage. Around this frequency
signal propagation is effected by oxygen molecule in the air. This phenomenon is 
called oxygen attenuation. Because such phenomenon, signal can not propagate for long distance.
In urban area where high bandwidth communication is required this property can increse
frequency reuse rate. Which will conserve frequency band resource.

\subsection{Calculations}
Free space loss is the loss when signal travels through open space with no
other attenuation. This can be calculated using following equation \cite{openspace}

\begin{equation}
    a_{sz}^{[dB]} = 32.44 + 20log f^{[MHz]}+20log d^{[km]} - G_{TX}^{[dB]} - G_{RX}^{[dB]}
    \label{eq:openspace}
\end{equation}

Oxygen attenuation or Atmospheric attenuation is the effect of signal propagation due to
gas in the atmosphere. A figure is given by 






%----------------------------------------------------------------------------------------
%	BIBLIOGRAPHY
%----------------------------------------------------------------------------------------

\renewcommand{\refname}{\spacedlowsmallcaps{References}} % For modifying the bibliography heading

\bibliographystyle{unsrt}

\bibliography{sample} % The file containing the bibliography

%----------------------------------------------------------------------------------------

\end{document}